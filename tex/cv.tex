\documentclass[11pt,letterpaper,roman]{moderncv}

\moderncvtheme[blue]{classic} \usepackage{sty/mycvconfig}

\firstname{\thefirstname{Arnaldo Ignacio}} \familyname{\\\thelastname{Gaspar
V\'ejar}}

\title{\normalfont \small  Licenciado en Ciencias de la Ingeniería,  Memorista
de Ingeniería Civil Informáticaa}

\address{Miravalle \#20037}{Pudahuel, Santiago -- Chile}

\mobile{56.9.8.814.90.77}                   \end{document}
\phone{56.2.2.601.46.75}                     \email{arnaldo.gaspar@ingennia.cl}
\homepage{github.com/arnaldog} \photo[64pt]{img/arnaldo.png}
%\quote{...}         


\begin{document} \maketitle

\section*{\textbf{Resumen}} \textsl Padre de un pequeño demonio desde cuatro
años y seis meses. Memorista de Ingeniería Civil Informática en la
\textsc{Universidad Técnica Federico Santa María}.  Actualmente realizando mi
trabajo de graduación en Series de Tiempo Caóticas de Alta Frecuencia usando
técnicas de Aprendizaje Computacional para predicción.  En el contexto laboral,
soy co fundador en \textsc{Ingennia} (www.ingennia.cl), con proyectos en curso
relacionados a las Aplicaciones Móbiles y la Geomática.

%	This curriculum vitae are designed as argument to be considerated in job
%	apply process for jr Javascript/Php Developer at Mcafee. The main
%	interest in apply to this job is to get strong knoledge in Node.js Api
%	design and MongoDb usage, and of course get embeed in a very serious
%	computer science engineering environment. 	
\section{Metas Laborales} \cvline{}{ Crecer integramente en la organización,
comprender, adquirir y propagar su visión, generar conocimientos  utilizando el
pensamiento sistémico e holístico.  Desarrollar y potenciar competencias de
trabajo en equipo, compartir conocimientos e ideas nuevas, profundizar y llevar
a la práctica temas técnicos de ingeniería. Colaborar con el bienestar y la
cultura organizacional promoviendo la identidad positiva y lograr posicionarme
en cargos relevantes en las areas técnicas y de gestión de proyectos.  }


\section{Antecedentes Laborales}
% Ingennia \cventry{December 2001} {Ingennia} % empresa {Fundador y Director} %
% cargo {area de ingenier\'ia y proyectos} % area {dirección de proyectos, pre
% venta} % desc

\subsection{Experiencia General en Ciencias de la Computación e Ingeniería de
Software} \cventry{desde Diciembre del 2011} {Ingennia} % empresa {Co Fundador}
% cargo {principalemente como ingeniero de proyectos y arquitecto
geo-espacial/móvil} {a cargo en ventas, gestión y desarrollo de software} {root
537, santiago, chile.}

% Multicaja
	\cventry{Invierno del 2011} {Multicaja S.A.} % Empresa {Software
Enginner} % cargo {Gerencia de Innovación, proyecto  \textit{``Multipay''}}
{Ingeniero de Software para aplicaciones transaccionales on-line.
Principalmente en el desarrollo de software Back-End.} {Phillips 56, Piso 5,
Santiago, Santiago, Chile.}

% Ikom
	\cventry{2010 -- 2011} {Ikom S.A.}	% empresa {Asesor en Ingeniería
de Software} {Departamento de Ingeniería Geo-Espacial
	%Sistemas de Informaci\'on Geogr\'aficos, proyectos 
	\textit{``Servidor de Mapas, Minera Los Pelambres''}y \textit{``Servidor
de Mapas Minera Collahuasi''}} {análisis y desarrollo para Sistemas de
Información Geográficos (SIG), procesamiento de cartografía digital, diseño e
implementación de metodologías de evaluación para nuevas funcionalidades SIG.
Elaboración de propuestas técnicas.} {Monse\~nor Sotero Sanz 55 Piso 5,
Providencia, Santiago, Chile.}
	

\cventry{Verano del 2009 -- 2010 \& Primavera del 2010} {Ikom S.A.}	%
empresa {Práctica Laboral \& Profesional} % cargo {Departamento de Ingeniería
Geo-Espacial, proyecto \textit{``Servidor de Mapa Minera Codelco El Teniente''}}
{Análisis y desarrollo de sistemas SIG, manipulación de cartografía digital,
evalución de desempeño y concurrencia para sistemas SIG de alto desempeño. }
{Monse\~nor Sotero Sanz 55 Piso 5, Providencia, Santiago, Chile.}
	
	

\subsection{Relatoría}
%	\cventry{Agosto 2012 - Octubre 2012} {Universidad Técnica Federico Santa
%	María} % Empresa {Co Relator} % cargo {Curso Taller de Desarollo de
%	Software} % Area {Co Relator en temas relativos a Mashups GIS } {Root
%	537, Santiago}
	
	\cventry{\small{Desde Septiembre a Octubre del 2012}} {Placetribe SpA} %
Empresa {Entrenamiento de Arquitectura y desarrollo de Software para producto}
{} % Area {MVC \& Mashup Architectural Style and Advanced Android Development
Trainning} {Agustinas 2356, Santiago, Chile.}


	\cventry{\small{Desde Julio to Agosto of 2012}} {Instituto Profesional
de Electrónica Gamma} % Empresa {Relator} % cargo {Curso SENCE} % Area {Linux
Avanzado} {Agustinas 2356, Santiago, Chile.}
	
	\cventry{\small{March of 2012}} {New Line Capacitación} % Empresa
{Relator} % cargo {Curso SENCE} % Area {Microsoft Office: Word and Excel}
{Realizado en SEGIC -- USACH, Avda. Lib. Bdo. O'Higgins \#3363, Est.  Central,
Santiago, Chile.}
	

	

	
	
\subsection{Projectos Comunitarios y Universitarios}

\cventry{Invierno del 2009} {Feria de Software 2009 at U.T.F.S.M.} {Software de
Evaluación de Competencias de Trabajo en Equipo} {Departamento de Informática}
{a cargo de diseñar y desarrollar un sistema de evaluación de competencias para
los equipos de la Feria de Software 2009} {Avenida Espa\~na 1680,  Valpara\'iso,
Chile.}

\cventry{Invierno del 2008} {Pre-Empresa Phoenix en la ``Feria de Software
2008'' de la U.T.F.S.M.} {Proyecto Motion Prisma: Sistema de Geo-Localización de
multimedios para Servicios turísticos} {Valparaíso, Chile}{} {\begin{itemize}
\item Encargado del análisis, modelado y desarrollo \end{itemize}}
	
%\cventry{Winter 2007} {Colegio Particular Playground} {Responsable Proyecto
%Servidor de Datos} {} {encargado de dise\~no, implmentaci\'on y despliegue de
%red de computadores con perfiles m\'oviles para alumnos y profesores en
%Intranet para la difusi\'on de material pedag\'ogico.} {Av Walker Mart\'inez
%3018, La Florida, Santiago.}
%

\cventry{Winter 2005} {Escuela Ernesto Quir\'oz Weber} {Profesor de
Alfabetización Digital} {Proyecto \textit{``Taller de Alfabetizaci\'on
Digital''}} {Profesor Voluntario para niños en riesgo social} {Avenida Matta 64,
Poblaci\'on Progreso, Valpara\'iso.}
 

%\cventry{Summer 2002} {Instituto Superior de Electr\'onica Gamma} {Asistente de
%Laboratorio de Armando y Mantenci\'on de Computadores} {Curso de Armado y
%Mantenci\'on de Computadores} {Asistente de apoyo para clases y pr\'acticas de
%armado y mantenci\'on de computadores y redes de computadores} {Agustinas 2356
%-- 2344, Santiago.}
\section{Antecedentes Académicos}
% university studies
\cventry{Education Universitaria} {\textsc{Universidad T\'ecnica Federico Santa
Mar\'ia (U.T.F.S.M)}} {Licenciatura en Ciencas de la Ingeniería}{Ingeniería
Civíl Informática, en trabajo de titulación} {} {Avenida España 1680,
Valparaíso.}
	
%\cventry{} {\textsc{Universidad T\'ecnica Federico Santa Mar\'ia}}
%{Ingenier\'ia Civil Inform\'atica}{ 2007 -- 2010} {Casa Central} {Valpara\'iso}
%\cventry{} {\textsc{Universidad T\'ecnica Federico Santa Mar\'ia}}
%{Ingenier\'ia Civil Telem\'atica}{ 2003 -- 2006 } {Casa Central} {Valpara\'iso}

% highschool studies
\cventry{Educación Media} {\textsc{Liceo de Aplicaci\'on}} {} {} {} {Avenida
Ricardo Cumming 21--29, Santiago.}
% primary studies
\cventry{Educación Básica} {\textsc{Escuela Cadete Arturo Prat Chac\'on}} {Ex
Escuela La Campana} {} {} {San Ignacio 196, Santiago.}

\subsection{Seminarios y Otros}

\cventry{Septiembre 2012} {Informática: Desafíos, herramientas y ciencia.}
{Escuela de Invierno 2012} {tópicos: Analítica del Aprendizaje: TIC's y Big Data
para la Investigación Educativa, Recuperación de Información en la Web: Espacios
Métricos y Diversidad de Resultados de Búsqueda. Procesamiento del Lenguaje
Natural (NLP): Semáticas Básicas para Aplicaciones} {Universidad T\'ecnica
Federico Santa Mar\'ia} {Avenida España 1680, Valpara\'iso.}
	
	
\cventry{Julio 2009} {High performance computing trends for 2010 and beyond}
{Final SCAT meeting \& Satellite School} {Scientific Computing Advanced
Training} {Universidad T\'ecnica Federico Santa Mar\'ia} {Avenida España 1680,
Valpara\'iso.}

%\cventry{Primavera 2001} {Curso Administrac\'on y mantenci\'on de Redes bajo
%Windows NT Server} {Instituto de Electr\'onica Gamma} {} {} {Santiago}
%	
%\cventry{Oto\~no 2001} {Administraci\'on y mantenci\'on de Redes para Trabajo
%en Grupo} {Instituto de Electr\'onica Gamma} {} {} {Santiago}
%	
%	
%
%	
\section{Resumen de Competencias Específicas} \subsection{Ingeniería de
Software} \cvline{Resumen de Experiencia y Conocimientos Generales} {Experiencia
en gestión de proyectos de software, espcialmente con adaptaciones de
metodologías ágiles. Expertiz en el proceso de preventa a través de propuestas
ténico económicas usando metodologías como \textsc{As Is-To Be}. Principalemente
envuelto en el desarrollo de SIG/RIA Mashups \& Desarrollo Móvil.}
	
\subsection{Competencias Computacionales} \cvline{Conocimientos en Computación
Científica} {Principalmente enfocado en Estadística Computacional (area de
interés), en temas relacionados al Análisis Inteligente de Datos (AIDA) y el
Aprendizaje Computacional (Machine Learning). Conocimientos relativo al modelado
de Redes Neuronales, Sistemas de Inferencia Difusos, Resampling Boostrap y
Simulación en genral. Además, conocimientos aplicados a Programación Entera y
Optimización Combinatoria a través de metaheurísticas como Algoritmos Genéticos
y Optimización de Enjambre de Partículas entre otros. Las principales
herramientas usadas al respecto son R-Project, Matlab/Octave y C++/Ansi-C
(OpenMP).}

\cvline{Arquitecturas de Web Services y SIG} {Experiencia como Arquitecto de
Aplicaciones usando el estilo RESTful tanto como en .Net Windows Communication
Foundation, Ruby on Rails, Python y Php. Integración Móvil/SIG a través de
Servicios Web estilo RESTful y componentes Geoespaciales como Postgres/Postgis y
Servicios Web SIG como WMS, WFS, etc.}
	

\cvline{Desarrollo Móvil} {Amplia experiencia en desarrollo Android con Android
SDK usando Maven y Spring Android.}

\cvline{Desarrollo Web Ágil} {Conocimientos amplios en el patrón de arquitectura
MVC con frameworks como Rails 3 (Ruby), Symfony 1.4 (php) y Java Server Faces
(JSF) principalmente.}


\cvline{Funcional y Scripting} {Amplia experiencia con Sencha Extjs4/3 para el
desarrollo de Aplicaciones de Internet Enriquecidas (RIA) junto con OpenLayers
para el desarrollo de Web SIG Mashups. }

\section{Disponibilidad} \cvline{Renta Esperada}{A convenir respecto a
responsabilidades.} \cvline{Disponibilidad}{Jornada Completa.}

\end{document}
