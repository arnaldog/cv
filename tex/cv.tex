\documentclass[11pt,letterpaper,roman]{moderncv}

\usepackage{xspace}
\moderncvtheme[blue]{classic} \usepackage{sty/mycvconfig}

\firstname{\thefirstname{Arnaldo Ignacio}} \familyname{\\\thelastname{Gaspar
V\'ejar}}

\title{\normalfont \small  Licenciado en Ciencias de la Ingeniería,  Egresado
de Ingeniería Civil Informática}

\address{Armando Moock \#3637, Depto 1003}{Macul, Santiago -- Chile}

\mobile{56.9.5.768.14.16}               
\phone{56.2.2.723.56.00}                     \email{arnaldog@gmail.com}
\homepage{linkd.in/1cUcfpF} \photo[64pt]{img/arnaldo.png}
%\quote{...}         


\begin{document} \maketitle

\section*{\textbf{Resumen}} \textsl Padre de un pequeño desde cuatro
años y tres meses. Egresado de Ingeniería Civil Informática en la
\textsc{Universidad Técnica Federico Santa María}. Soy co fundador en
\textsc{Ingennia} (www.ingennia.cl), con proyectos en curso relacionados a las
Aplicaciones Móviles y la Geomática.

%	This curriculum vitae are designed as argument to be considerated in job
%	apply process for jr Javascript/Php Developer at Mcafee. The main
%	interest in apply to this job is to get strong knoledge in Node.js Api
%	design and MongoDb usage, and of course get embeed in a very serious
%	computer science engineering environment. 	
\section{Metas Laborales} \cvline{}{ Crecer integramente en la organización,
comprender, adquirir y propagar su visión. Actualmente estoy muy interesado en
integrarme a nuevos y exigentes grupos de trabajo, trabajando duro en nuevos
problemas tecnológicos, compartiendo conocimientos e ideas relativas a las
ciencias de la computación e ingeniería de software.}

\section{Antecedentes Académicos}
% university studies
\cventry{Educación Universitaria} {\textsc{Universidad T\'ecnica Federico Santa
Mar\'ia (U.T.F.S.M)}} {Licenciatura en Ciencas de la Ingeniería}{Ingeniería
Civíl Informática, en trabajo de titulación} {} {Avenida España 1680,
Valparaíso.}
	
%\cventry{} {\textsc{Universidad T\'ecnica Federico Santa Mar\'ia}}
%{Ingenier\'ia Civil Inform\'atica}{ 2007 -- 2010} {Casa Central} {Valpara\'iso}
%\cventry{} {\textsc{Universidad T\'ecnica Federico Santa Mar\'ia}}
%{Ingenier\'ia Civil Telem\'atica}{ 2003 -- 2006 } {Casa Central} {Valpara\'iso}

% highschool studies
\cventry{Educación Media} {\textsc{Liceo de Aplicaci\'on}} {} {} {} {Avenida
Ricardo Cumming 21--29, Santiago.}
% primary studies
\cventry{Educación Básica} {\textsc{Escuela Cadete Arturo Prat Chac\'on}} {Ex
Escuela La Campana} {} {} {San Ignacio 196, Santiago.}

\subsection{Seminarios y Otros}

\cventry{Septiembre 2012} {Informática: Desafíos, herramientas y ciencia.}
{Escuela de Invierno 2012} {tópicos: Analítica del Aprendizaje: TIC's y Big Data
para la Investigación Educativa, Recuperación de Información en la Web: Espacios
Métricos y Diversidad de Resultados de Búsqueda. Procesamiento del Lenguaje
Natural (NLP): Semáticas Básicas para Aplicaciones} {Universidad T\'ecnica
Federico Santa Mar\'ia} {Avenida España 1680, Valpara\'iso.}
	
	
\cventry{Julio 2009} {High performance computing trends for 2010 and beyond}
{Final SCAT meeting \& Satellite School} {Scientific Computing Advanced
Training} {Universidad T\'ecnica Federico Santa Mar\'ia} {Avenida España 1680,
Valpara\'iso.}


\section{Antecedentes Laborales}
% Ingennia \cventry{December 2001} {Ingennia} % empresa {Fundador y Director} %
% cargo {area de ingenier\'ia y proyectos} % area {dirección de proyectos, pre
% venta} % desc



\subsection{Experiencia General en Ciencias de la Computación e Ingeniería de
Software}

\cventry{10/2013-- Actualidad} {\se} {\instagisSPA} {\stgo} {}
{Ingeniero de Software en \instagis. \\ Rosario Norte 555 Oficina 2102. Las Condes, Santiago.}


\cventry{7/2013--10/2013} {\pe} {\mapcity} {\stgo} {}
{IArquitecto de Software y Programador. Diseño e implementación de \mientorno   y proyectos internos. \\ San Sebastián 2952, piso 3. Las Condes, Santiago.}

\cventry{4/2013--7/2013} {\scd} {\paonde} {\stgo} {}
{Servicio de consultoría en diseño y desarrollo para la aplicación paOnde en
su versión Android.}

\cventry{4/2013--7/2013} {\scd} {\paonde} {\stgo} {}
{Servicio de consultoría en diseño y desarrollo para la aplicación paOnde en
su versión Android. \\ Matías Castro, matias.castro@paonde.com}

\cventry{7/2012--9/2012} {\tchr} {\placetribe} {\stgo} {}
{Consultoría en desarrollo Android para Placetribe. Entrenamiento en JAVA OO,
Patrones de Diseño, Android SDK e integración REST con Android. Además, una
breve reseña de patrones y estilos de arquitectura en el contexto del desarrollo
de las aplicaciones Placetribe. \\ Tomás Arredondo Vidal, +56322654212, tomas.arredondo@usm.cl}


\cventry{12/2011--Presente} {\cf} {\ingennia} {\stgo} {}
{Principalmente concentrado en la investigación y desarrollo de nuevos productos
de software como las aplicaciones móviles y el software como servicio. En
Ingennia integramos tecnología geo espacial, web y móvil para crear soluciones
escalables, de calidad y con grandes características. \\ Root 537, Santigo.}

       % LET   
\cventry{12/2011--9/2012} {\scd} {\jobbitgames} {\stgo} {}
{Servicios de consultoría en arquitectura de software GIS y desarrollo para el
``Sistema de Levantamiento en Terreno'' (LET) para Minera la Escondida (MEL).
\\ Enrique Tietzen, etietzen@jobbitgames.com}


 % MATAVERI
\cventry{10/2011--4/2012} {\scd} {\mataveri} {\stgo} {}
{Servicios de consultoría y desarrollo para servicios de información turísticos en
plataformas Android, iPhone y software Backend a través de las aplicaciones
\freetouchchile y \freetouchperu. Freetouch integra servicios web Rest y
sincroniza información (imágenes, localización y descripciones) de servicios
turísticos. \\ 
Gonzalo Fuenzalida F. gff@mataveri.com \\ El Golf 192, Piso 14, Santiago.}

  % Multicaja
\cventry{6/2011--9/2011} {\se} {\multicaja} {\stgo} {}
{Desarrollador Ruby on Rails para el equipo de Innovación. Principales tareas
son las de programar software de backend y mantener y modificar software
existente. \\ Nicolás Mery, nicolas.mery@multicaja.cl \\ Phillips 56, Piso 5, Santiago.}


% Ikom
\cventry{5/2010--5/2011} {\sd} {\ikom} {\stgo} {}
{Trabajó en el departamento de ingeniería geo espacial en el desarrollo del
servidor de mapas corporativo (SDM), el cual fue puesto en marcha para las
mineras Los Pelambres y Collahuasi. El desarrollo se realiza principalmente
sobre PostGis, Windows Communication Foundation (WCF), Openlayers y ExtJS 3
(Javascript) \\ Martina Mann. martina.mann@gmail.com \\ Monseñor Sotero Sanz 55 Piso 5, Providencia, Santiago.} 


 \cventry{12/2009--4/2010} {\intership} {\ikom} {\stgo} {}
{Desarrollo de funcionalidades GIS para el Servidor de Mapas Corporativo (SDM)
para la minera Codelco El Teniente. Análisis de desempeño para concurrencia en
servidores de mapa. \\ Desarrollo GIS utilizando PostGis, Geoserver, WCF,
Openlayers y ExtJS3. \\ Martina Mann. martina.mann@gmail.com. Monseñor Sotero Sanz 55 Piso 5, Providencia, Santiago.}

	
\subsection{Proyectosa Comunitarios y Universitarios}
%	\cventry{Agosto 2012 - Octubre 2012} {Universidad Técnica Federico Santa
%	María} % Empresa {Co Relator} % cargo {Curso Taller de Desarollo de
%	Software} % Area {Co Relator en temas relativos a Mashups GIS } {Root
%	537, Santiago}

\cventry{7/2013--8/2013} {\sd} {\ingennia} {\stgo} {}
{Desarrollo de Aplicaciòn \vocablia, una aplicación gratuita para smartphones que ayuda a mejorar el léxico español. Disponible en AppStore y Google Play.}


\cventry{7/2012--8/2012} {\tchr} {\ipgamma} {\stgo} {}
{Curso \sence Linux Avanzado en tópicos de Networking, administración de
recursos, Bash scripting y edición de texto. \\ Agustinas 2356, Santiago.}

% NEW LINE
\cventry{3/2012} {\tchr} {\otecnewline} {\stgo} {}
{Curso \sence en Microsoft Word y Excel orientado al uso de tablas dinámicas y
consultas de bases de datos. \\ Realizado en SEGIC/USACH, Avda. Lib. Bdo. O'Higgins \#3363, Est. Central, Santiago, Chile.}

\cventry{2011}{\se}{Allizia}{Santiago, CH}{}
{Allizia es una librería de metaheurísticas escrita en C++ con OpenMP (en
construcción) cuyo objetivo es el de implementar problemas de la Inteligencia
Artificial (IA) de forma fácil y flexible. }

\cventry{Invierno 2009}{\sd} {\fdsw{2009} } {} {}
{Desarrollador del software de evaluación de competencias de trabajo en equipo.}



\cventry{Invierno 2008} {\sd} {\fdsw{2008} } {} {}
{Desarrollador y analista para el proyecto Motion Prisma de la Pre empresa
Phoenix de la \fdsw{2008}, un sistema de multimedios georeferenciados con fines
turísticos.}

\cventry{Invierno 2005} {\tchr} {\ernestoquiroz } {\valpo} {}
{Volutario del proyecto \textit{``Taller de Alfabetizaci\'on Digital''}
para niños en riesgo social. \\ Avenida Matta 64, Población el Progreso,
Valparaíso.}
	
	
%\cventry{Winter 2007} {Colegio Particular Playground} {Responsable Proyecto
%Servidor de Datos} {} {encargado de dise\~no, implmentaci\'on y despliegue de
%red de computadores con perfiles m\'oviles para alumnos y profesores en
%Intranet para la difusi\'on de material pedag\'ogico.} {Av Walker Mart\'inez
%3018, La Florida, Santiago.}
%

%\cventry{Summer 2002} {Instituto Superior de Electr\'onica Gamma} {Asistente de
%Laboratorio de Armando y Mantenci\'on de Computadores} {Curso de Armado y
%Mantenci\'on de Computadores} {Asistente de apoyo para clases y pr\'acticas de
%armado y mantenci\'on de computadores y redes de computadores} {Agustinas 2356
%-- 2344, Santiago.}

%\cventry{Primavera 2001} {Curso Administrac\'on y mantenci\'on de Redes bajo
%Windows NT Server} {Instituto de Electr\'onica Gamma} {} {} {Santiago}
%	
%\cventry{Oto\~no 2001} {Administraci\'on y mantenci\'on de Redes para Trabajo
%en Grupo} {Instituto de Electr\'onica Gamma} {} {} {Santiago}
%	
%	
%
%	
\section{Resumen de Competencias Específicas} \subsection{Ingeniería de
Software} \cvline{Resumen de Experiencia y Conocimientos Generales} {Experiencia
en gestión de proyectos de software, espcialmente con adaptaciones de
metodologías ágiles. Expertiz en el proceso de preventa a través de propuestas
ténico económicas usando metodologías como \textsc{As Is-To Be}. Principalemente
envuelto en el desarrollo de SIG/RIA Mashups \& Desarrollo Móvil.}
	

\cvline{Arquitecturas de Web Services y SIG} {Experiencia como Arquitecto de
Aplicaciones usando el estilo RESTful tanto como en .Net Windows Communication
Foundation, Ruby on Rails, Python y Php. Integración Móvil/SIG a través de
Servicios Web estilo RESTful y componentes Geoespaciales como Postgres/Postgis y
Servicios Web SIG como WMS, WFS, etc.}
	

\cvline{Desarrollo Móvil} {Amplia experiencia en desarrollo Android con Android
SDK usando Maven y Spring Android.}

\cvline{Desarrollo Web Ágil} {Conocimientos amplios en el patrón de arquitectura
MVC con frameworks como Rails 3 (Ruby), Symfony 1.4 (php) y Java Server Faces
(JSF) principalmente.}


\cvline{Funcional y Scripting} {Amplia experiencia con Sencha Extjs4/3 para el
desarrollo de Aplicaciones de Internet Enriquecidas (RIA) junto con OpenLayers
para el desarrollo de Web SIG Mashups. }

\subsection{Competencias Computacionales} \cvline{Conocimientos en Computación
Científica} {Principalmente enfocado en Estadística Computacional (area de
interés), en temas relacionados al Análisis Inteligente de Datos (AIDA) y el
Aprendizaje Computacional (Machine Learning). Conocimientos relativo al modelado
de Redes Neuronales, Sistemas de Inferencia Difusos, Resampling Boostrap y
Simulación en genral. Además, conocimientos aplicados a Programación Entera y
Optimización Combinatoria a través de metaheurísticas como Algoritmos Genéticos
y Optimización de Enjambre de Partículas entre otros. Las principales
herramientas usadas al respecto son R-Project, Matlab/Octave y C++/Ansi-C
(OpenMP).}

\section{Disponibilidad} \cvline{Renta Esperada}{A convenir respecto a
responsabilidades.} \cvline{Disponibilidad}{Jornada Completa.}

\end{document}
